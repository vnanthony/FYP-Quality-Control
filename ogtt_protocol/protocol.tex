\documentclass{article}
\usepackage[a4paper, top=1in, bottom=1in, total={6in, 8in}]{geometry}

\title{Experiment Protocol of the Project Non-invasive Blood Glucose Measurement using Photoplethysmography and AI Machine Learning Algorithm}
\date{Nov 2023}


\begin{document}

\maketitle

\section{Introduction}
Diabetes is a chronic disease that affects more than 400 million people worldwide, according to the World Health Organization.
The blood glucose meter market is expected to grow from USD 7,419.0 Million in 2018 to USD 15,415.6 Million by 2026, with a compound annual growth rate (CAGR) of 9.6\% \cite{noauthor_blood_2019}.
However, current blood glucose meters rely on invasive blood tests, which may discourage frequent monitoring and fail to provide real-time feedback.
Therefore, there is an urgent need for a non-invasive blood glucose monitoring device to enhance diabetes management.

Photoplethysmography (PPG) is a promising technique for non-invasive blood glucose measurement, as it can estimate biological parameters from optical signals and has low cost \cite{vargova_estimation_2023}.
The common approach for using PPG to measure blood glucose level (BGL) is to extract features from PPG signals and then apply machine learning models to estimate the BGL.
However, most of the existing methods do not meet the clinical requirement, which states that at least 99\% of the results must be within zones A and B of the consensus error grid analysis.

In this study, we propose a novel machine learning algorithm for non-invasive blood glucose measurement using PPG signals.
We use the largest public dataset (MIMICIII) to train and test our algorithm.
We also develop a signal processing and feature extraction pipeline to improve the accuracy of the BGL estimation.
Our results show that our algorithm achieves 93.6\% accuracy, with 96.15\% of the data in zone A and 3.85\% of the data in zone B.
This is an improvement over the state-of-the-art methods. We are currently working on optimizing our algorithm and designing a specific device for glucose measurement.
We believe that our method has the potential to reach the clinical accuracy and benefit millions of diabetes patients.

\section{Objectives}
To evaluate the blood glucose level detection capability of the PPG approach, a study is required to compare the PPG method and the traditional invasive blood glucose testing method.
Since this study mainly focuses on the blood glucose level of healthy people as a proof of concept, only healthy subjects will be recruited.

\section{Data Delivery}
\begin{enumerate}
\item The answers of personal attributes
\item The heart rate in bpm
    \begin{enumerate}
        \item at time 0 min (blood glucose baseline)
        \item at time 30 mins
    \end{enumerate}
\item The 8-minute PPG signal
    \begin{enumerate}
        \item at time 0 min (blood glucose baseline)
        \item at time 30 mins
    \end{enumerate}
\item Blood glucose level
    \begin{enumerate}
        \item at time 0 min (blood glucose baseline)
        \item at time 30 mins
    \end{enumerate}
\end{enumerate}

\section{Study Protocol}
This study will be conducted according to the Declaration of Helsinki's ethical principles for medical research on human subjects.
Informed consent will be obtained from each participating subject. This protocol refers to the Endocrinology Handbook published
by Imperial College Healthcare NHS Trust \cite{ali_endocrinology_2018}.

\subsection{Selection of the subjects}
\subsubsection{Source of subjects}
COCHE, Hong Kong
\subsubsection{Target number of subjects}
50

\subsubsection{Inclusion Criteria}
\begin{itemize}
\item Healthy people between 18 and 65 years old
\end{itemize}

\subsubsection{Exclusion Criteria}
\begin{itemize}
\item Patients with implantable cardiac devices, including permanent pacemakers, cardiac-resynchronization therapy or defibrillator
\item Pregnant people
\item People who are unable to sign the informed consent

\end{itemize}

\section{Study Workflow}

\subsection{Personal attributes}
\begin{enumerate}
    \item Age
    \item Gender
    \item Height (cm)
    \item Weight (kg)
    \item Time elapsed from the last meal (hours)
    \item Ethnicity
    \begin{itemize}
        \item White (white)
        \item East Asian (east\_asian)
        \item West Asian (west\_asian)
        \item Middle Asian (middle\_asian)
        \item South Asian (south\_asian)
        \item Southeast Asian (southeast\_asian)
        \item African Americans (african\_americans)
        \item Hispanic and Latino Americans (hispanic\_and\_latino\_americans)
        \item Caribbean or Black (caribbean\_or\_black)
        \item Other (other)
    \end{itemize}
    \item Diabetes situation
    \begin{itemize}
        \item No diabetes (no)
        \item Type 1 diabetes (type\_1)
        \item Type 2 diabetes (type\_2)
    \end{itemize}
    \item The total number of cigarettes consumed from this time last year to the present moment
    \begin{itemize}
        \item 0 (a)
        \item 1-50 (b)
        \item >50 (c)
    \end{itemize}
    \item The monthly average number of times alcohol was consumed from this time last year to the present moment
    \begin{itemize}
        \item 0 (a)
        \item 1-2 (b)
        \item >2 (c)
    \end{itemize}
\end{enumerate}

\subsection{Preparation of the Subject}
\begin{itemize}
    \item Fast overnight such that there is no food intake for at least \textbf{six} hours.
\end{itemize}
\subsection{Preparation of the Researcher}
\subsubsection{On the Day before the Experiment}
\begin{enumerate}
    \item Prepare oral glucose load
        \begin{enumerate}
        \item Buy bottles of original-taste Lucozade (70 kCal/100mL formulation)
        \end{enumerate}
\end{enumerate}
\subsubsection{On the Day of the Experiment}
\begin{enumerate}
\item Prepare a quiet and relaxing environment without ambient noise.
\item Use a measuring cup to prepare \underline{440 mL} of Lucozade drink in room temperature.
\item Prepare the Contour plus ELITE blood glucose monitor
    \begin{itemize}
    \item Test paper.
    \item Replaceable needles.
    \end{itemize}
\item Prepare the alcohol wiper for sterilization.
\item Prepare the Finapres device.
    \begin{enumerate}
    \item Connect the SpO2 sensor to the Finapres device.
    \item Switch on the Finapres device.
    \item Press 'Setup and start measurement'.
    \item Press 'Continue'.
    \item Press 'Reload patient data'.
    \item Press 'Continue'.
    \end{enumerate}
\item Prepare the Analog Devices PPG device.
    \begin{enumerate}
    \item Connect the ADPD4200 motherboard to the computer.
    \item Switch on the motherboard. Two LED lights should be ON.
    \item Connect the AFE board to the motherboard.
    \item Open the Wavetool Evaluation software.
    \item Load the configuration.
    \item Open C, D and E slots.
    \end{enumerate}
\item Prepare a stopwatch.
\item Prepare a timer and set the time to 8 minutes.
\item Prepare tissues.
\item Open the excel file from the folder 'ideation\_lester/ogtt' for recording the personal attributes,\
the blood glucose level and the PPG file name.
\end{enumerate}

\subsection{Procedures}
\begin{enumerate}
\item Invite a subject who has fasted overnight.
\item Ask the subject to sign the consent form.
\item Ask the subject to fill in the diabetes questionnaire.
\item Ask the subject for his/her personal attributes and record the answers
\item Ask the subject to sit comfortably on a chair in the prepared environment.
\item Ask the subject to place the left arm onto the table in a comfortable position.
\item Read the blood glucose level with the blood glucose monitor.
    \begin{enumerate}
    \item Sterilize the whole blood glucose monitor setup with alcohol.
    \item Place a new replaceable needle into the cartridge of the finger puncturing tool.
    \item Sterilize the right-hand index fingertip with alcohol.
    \item Insert the test paper into the blood glucose monitor.
    \item Ask the subject to use the finger puncturing tool to puncture the right-hand index fingertip.


    \item Ask the subject to absorb the blood with the blood glucose test paper.
    \item Collect the used puncturing tool as a medical waste.
    \item Provide the subject with tissues to cover the wound on the fingertip.
    \item Read and record the blood glucose level to the excel file.
    \end{enumerate}
\item Use the Finapres device to get the heart rate in bpm.
    \begin{enumerate}
    \item Clip the SpO2 sensor onto the subject's left index finger.
    \item When there is a stable heart rate, record it to the excel file.
    \item Remove the SpO2 sensor from the subject's left index finger.
    \end{enumerate}
\item Use the Analog Devices device to get the PPG signal.
    \begin{enumerate}
    \item Fix the PPG sensor onto the subject's thumb with the Finapres clip and a black velcro tape.
    \item Start logging the PPG signals.
    \item Start the timer.
    \item Stop recording the PPG signals when the time is up.
    \item Save the PPG signals and record the file name in the excel file.
    \item Remove the PPG sensor from the subject's left hand.
    \item Move the logged data folder to `ideation\_lester/ogtt/ppg'.
    \end{enumerate}
\item If the blood glucose level is higher than 6.9 mmol/L, test the level again \cite{noauthor_indicator_nodate}.
\item If the level is still higher than 6.9 mmol/L, stop the experiment at this point.
Suggest the subject that he/she should seek professional medical consultation.
\item Ask the subject to drink the 440 mL of Lucozade drink within 5 minutes.
\item Start the stopwatch (time = 0 min).
\item Repeat steps 6-9 when the stopwatch shows 30 mins.
\end{enumerate}

\bibliographystyle{plain}
\bibliography{protocol}

\end{document}